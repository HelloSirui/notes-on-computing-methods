\chapter{数值积分与数值微分}

\entry 常规(解析)积分的\emph{局限性}:往往难以求得原函数,无法应用 Newton-Leibniz 公式;有时
仅知函数在个别离散点的取值(列表函数)。

\entry \emph{数值积分}/\emph{数值微分}的共同思路:用个别点处\emph{函数值 $f(x_i)$ 的线性组合},来估计整个区
间上的积分值或个别点附近的微分值。
\begin{gather}
\int_a^bf(x)\di x=\sum_{i=0}^nA_if(x_i)\\
f'(x)=\sum_{i=0}^nB_if(x_i)
\end{gather}

\entry 主要问题:节点$x_i$的选取;求积系数$A_k$的计算;误差估计与评价。

\entry 记号:对于$f(x)$,记其准确积分值为 $I[f]=\int_a^bf(x)\di x$,而记其数值估计公式为
\[Q[f]=\sum_{i=0}^nA_if(x_i)\approx I[f].\]
并将数值积分公式的误差记为$R[f]=I[f]-Q[f]$.

\section{插值型积分及其延伸}
\entry\label{6-et1} 两种近似积分思路:
\begin{enumerate}
    \item 按 Riemann 积分定义,作区间的划分,按若干小矩形估算:
    \[\int_a^bf(x)\di x\approx\sum_{i=o}^nf(x_i)(x_{i+1}-x_i)\]
    \item 取若干数据点插值,用插值多项式$p(x)$的积分估计函数的积分:
    \[\begin{aligned}
    \int_a^bf(x)\di x&\approx\int_a^bp(x)\di x
    =\int_a^b\left[\sum_{i=0}^nl_i(x)f(x_i)\right]\di x\\
    &=\sum_{i=0}^n\left(\int_a^bl_i(x)\di x\right)\cdot f(x_i).
    \end{aligned}\]
\end{enumerate}
共同特点是:用若干点上函数值$f(x_i)$之\emph{线性组合}估计整个区间上的积分值。问题:无法估计
误差。应当从此种思路出发,直接构造估计公式,进而就可估计误差。

\entry \key{Newton-Cotes 型求积公式}:在区间上等距取定若干\emph{插值点},构造\emph{插值多项式}并以其积分代替$f(x)$的积分。

\entry 记号:设节点数为$n+1$个:$x_0,x_1,\cdots,x_n$(一般常取$n=1,2,4$),记此时各点的距离为$h=\frac{b-a}n$。

\entry \key{梯形公式}:当$n=1$时,取定区间的左右两端点为$x_0$与$x_1$,可得到
\begin{equation}
Q_1[f]=\frac{b-a}2\left[f(a)+f(b)\right].
\end{equation}
此即\emph{梯形求积公式},常将$Q_1[f]$记为$T_1$。由第一积分中值定理
\footnote{设$f,g\in C[a,b]$且$g$在$[a,b]$不变号,则存在$\eta\in[a,b]$使
$\int_a^bf(x)g(x)\di x=f(\eta)\int_a^bg(x)\di x$。}
可估计误差为
\begin{equation}
R_1[f]=\frac{(b-a)^3}{12}f''(\eta),\,\eta\in(a,b)
\end{equation}

\entry \key{Simpson 公式}:当$n=2$时,$h=\frac{b-a}2$,将取得以下等距节点:
\[x_0=a,\,x_1=\frac{a+b}2,\,x_2=b.\]
作插值多项式并积分,可得求积公式为
\begin{equation}
Q_2[f]=\frac h3\left[f(a)+4f\left(\frac{a+b}2\right)+f(b).\right]
\end{equation}
此即 \emph{Simpson 公式},常将$Q_2[f]$简记为$S_1$。利用之后的方法
\footnote{即广义 Peano 定理与「24K金法」。}
可估计误差为
\begin{equation}
R_2[f]=-\frac{(b-a)^5}{2880}f^{(4)}(\eta).
\end{equation}

\entry \key{Cotes 求积公式}:当$n=4$时,$h=\frac{b-a}4$,插值并积分可得
\begin{equation}
Q_4[f]=\frac{b-a}{90}[7f(a)+32f(a+h)+12f(a+2h)+32f(a+3h)+7f(b).]
\end{equation}
此即 \emph{Cotes 求积公式},常将$Q_4[f]$简记为$C_1$。利用之后的方法可估计误差为
\begin{equation}
R_4[f]=-\frac{(b-a)^7}{1935360}f^{(6)}(\eta).
\end{equation}

\entry \key{代数精度}:设有数值积分公式,且对于不超过$m$次的多项式$f$均
有$R[f]=0$成立,而对某$m+1$次多项式$f$即有$R[f]\neq0$,则称该数值积分公式的代数精度为$m$。

\entry 对\emph{梯形公式},其代数精度为 $1$;对 \emph{Simpson 公式},其代数精度为 $2$;对 \emph{Cotes 公式},
其代数精度为 $4$。

\entry 不宜采用次数过高的多项式插值:\emph{Runge 现象}
\footnote{且根据后面的内容可知,次数过高的 Newton-Cotes 公式稳定性差,结果易发散}。

\entry \key{复化求积公式}:对每个小区间 $[x_{k-1},x_k]$ 分别运用积分公式,再将结果求和。

\entry 公式表述:对由 $n+1$ 个数据点所划分的 $n$ 个小区间,对以上三种插值型积分公式的
结果如下。
\begin{itemize}
    \item 取 $n=1$,得\key{复化梯形公式}:
    \begin{equation}
    T_n=\frac h2\left[f(a)+2\dsum_{i=1}^{n-1}f(x_i)+f(b)\right],
    \end{equation}
    其误差估计为 $R_{T_n}=-\frac{b-a}{12}h^2f''(\eta)$。
    \item 取 $n=2$,得\key{复化 Simpson 公式}:
    \begin{equation}
    S_n=\frac h2\left[f(a)+2\dsum_{i=1}^{n-1}f(x_i)+f(b)\right],
    \end{equation}
    其误差估计为 $R_{S_n}=-\frac{b-a}{2880}h^4f^{(4)}(\eta)$。
    \item 取 $n=4$,得\key{复化 Cotes 公式}:
    \begin{equation}\begin{aligned}
    C_n=&\frac h{90}\left\{7f(a)+32\dsum_{i=0}^{n-1}\left[
    f\left(x_i+\frac h4\right)+f\left(x_i+\frac{3h}4\right)\right]+\right.\\
    &\left.12\dsum_{i=0}^{n-1}f\left(x_i+\frac h2\right)+14\dsum_{i=1}^{n-1}f(x_i)
    +7f(b)\right\}.
    \end{aligned}\end{equation}
    其误差估计为 $R_{c_n}=-\frac{b-a}{1935360}h^6f^{(6)}(\eta)$。
\end{itemize}

\entry \key{变步长积分法}:为进一步减小误差,考虑在用复化求积公式时不断缩小步长。为此,先凑出一个近似的误差估计式(减小步长的依据)。设用复化梯形公式 $T_n$ 求积分,当取得 $n+1$ 个点时的误差估计式为
\begin{equation}
R_n[f]=I[f]-T_n=-\frac{b-a}{12}h^2f''(\eta_1)
\end{equation}
将子区间数量翻倍,则取得 $2n+1$ 个点时的误差估计式为
\begin{equation}
R_{2n}[f]=I[f]-T_{2n}=-\frac{b-a}{12}\left(\frac h2\right)^2f''(\eta_2)
\end{equation}
若估计$f''(\eta_1)\approx f''(\eta_2)$,则联立两式即可得到
\begin{equation}\label{6-e1}
R_{2n}[f]=I[f]-T_{2n}\approx\frac13(T_{2n}-T_n)
\end{equation}
即积分误差可以用变步长结果间的差距 $T_{2n}-T_n$ 所估计。因此,若希望实现 $R[f]<\vepsilon$,可按如下步骤实施变步长复化积分:
\begin{enumerate}\tl
    \item $n=1$,$h=b-a$,计算$T_n$;
    \item 缩小步长$h$至原来的一半,计算$T_{2n}$;
    \item 验证误差估计条件 $|T_{2n}-T_n|<\vepsilon$ 是否满足,若满足则停止,若不满足则重复上一步。
\end{enumerate}

\entry $T_{2n}$的迭代算法:
\begin{equation}
T_{2n}=\frac12T_n+\frac h2\sum_{k=1}^nf\left(\frac{x_{k-1}+x_k}2\right)
\end{equation}

\entry 变步长积分公式中,同样可采用复化 Simpson 公式或复化 Cotes 公式:
\begin{gather}
I-S_{2n}\approx\frac1{4^2-1}(S_{2n}-S_n),\\
I-C_{2n}\approx\frac1{4^3-1}(C_{2n}-C_n).
\end{gather}

\entry 启发:由误差估计式 (\ref{6-e1}) 可以猜想,以下的估计式同样成立:
\[ I\approx T_{2n}+\frac1{4-1}(T_{2n}-T_n) \]
而将 $T_n$ 与 $T_{2n}$ 的表达式代入之后「竟然」发现
\begin{equation}
T_{2n}+\frac1{4-1}(T_{2n}-T_n)=S_n
\end{equation}
由此说明,\emph{可用低精度公式组合出高精度公式}
\footnote{本质上是由插值点的增加所致。}
。对复化 Simpson 公式实行类似操作可推得
\begin{equation}
S_{2n}+\frac1{4^2-1}(S_{2n}-S_n)=\frac{4^2S_{2n}-S_n}{4^2-1}=C_n
\end{equation}


\entry \key{Romberg 积分}:进一步考虑对 Cotes 公式的结果,将得到一个新的积分公式:
\begin{equation}
C_{2n}+\frac1{4^3-1}(C_{2n}-C_n)=\frac{4^3C_{2n}-C_n}{4^3-1}=R_n
\end{equation}
称由上式表述的 $R_n$ 为 \key{Romberg 积分},其代数精度 $m=7$,误差估计式为
\begin{equation}
R[f]=K\cdot h^8f^{(8)}(\eta).
\end{equation}

\entry 一般不在 Romberg 积分的基础上进一步递推,因被积函数的高阶导数性态不易估计,仍可能出现 Runge 现象;且$R_{2n}-R_n$ 一项很小,舍入误差较大;计算量也过大。

\entry Romberg 积分可由最基本的复化梯形公式逐阶递推而来,常用 \key{Romberg 计算图}辅助
计算。

\begin{figure}[htbp]
\small\renewcommand{\arraystretch}{0.7}\centering
\begin{tabular}{ccccccc}
\toprule
$T$ && $S$ && $C$ && $R$ \\
\midrule
$T_1$ && && && \\
 & $\searrow$ & && && \\
$T_2$ & $\rightarrow$ & $S_1$ && &&\\ 
 & $\searrow$ & & $\searrow$ & && \\
$T_3$ & $\rightarrow$ & $S_2$ & $\rightarrow$ & $C_1$ && \\
& $\searrow$ & & $\searrow$ & & $\searrow$ & \\
$T_4$ & $\rightarrow$ & $S_3$ & $\rightarrow$ & $C_2$ & $\rightarrow$ & $R_1$ \\
& $\searrow$ & & $\searrow$ & & $\searrow$ & \\
$T_5$ & $\rightarrow$ & $S_4$ & $\rightarrow$ & $C_3$ & $\rightarrow$ & $R_2$ \\
$\vdots$ && $\vdots$ && $\vdots$ && $\vdots$ \\
\bottomrule
\end{tabular}
\caption{Romberg 计算图}\label{6-f1}
\end{figure}

\section{待定系数法与 Gauss 型积分}
\trm 积分公式 $Q[f]$ 之\emph{代数精度}为 $m$ 的充要条件是
\[R[x^k]=0,\,(k=0,1,2,\cdots,m),\,R[x^{m+1}]\neq0.\]
可利用此结果,在积分公式中分别代入$f(x)=x^k$,验算积分公式的代数精度。

\entry \key{待定系数法}:根据数值积分公式的一般形式 $Q[f]=\dsum_{i=0}^kA_if(x_i)$,
在给定条件下(如达到指定的代数精度、给定已知节点),代入若干不同的 $f(x)$,求解出公式中未知的节点$x_i$与系数$A_i$。

\example 待定系数法导出梯形公式:设 $Q[f]=A_1f(a)+A_2f(b)$,希望其代数精度为 $m=1$,则有
\begin{itemize}
    \item 令 $f(x)=1$,由 $I[f]=Q[f]$ 有 $A_1-A_2=b-a$;
    \item 令 $f(x)=x$,由 $I[f]=Q[f]$ 有 $A_1\cdot a + A_2\cdot b=\frac12(b^2-a^2)$。
\end{itemize}
解得 $A_1=A_2=\frac{b-a}2$。故积分公式为
\[Q[f]=\frac{b-a}2[f(a)+f(b)],\]
此即梯形公式。

\entry \key{广义 Peano 定理}:设 $Q[f]$ 的截断误差 $R[f]$是区间上 $m+1$ 阶导数连续的函数
之线性泛函\footnote{线性泛函满足:$R[c_1f_1+c_2f_2]=c_1R[f_1]+c_2R[f_2]$。},且其代数精度为 $m$,则有 $R[f(x)]=R[e(x)]$,其中
\begin{equation}\label{6-e2}
e(x)=\frac{f^{(m+1)}(\xi)}{(m+1)!}(x-\tilde{x}_0)(x-\tilde{x}_1)\cdots
(x-\tilde{x}_m)
\end{equation}
$\tilde{x}_0,\tilde{x}_1,\cdots,\tilde{x}_m$ 为 $[a,b]$ 上任意一点,$\xi\in[a,b]$
与这 $m+1$ 个点的选取有关。

\entry 通过合适的\emph{取点},可由广义 Peano 定理求得各积分公式的误差估计。一般要求:
\begin{itemize}\tl
    \item $e(x)$表达式中的 $(x-\tilde{x}_0)(x-\tilde{x}_1)\cdots(x-\tilde{x}_m)$ 项在 $[a,b]$ 不变号;
    \item 选取 $\tilde{x}_0,\tilde{x}_1,\cdots,\tilde{x}_m$ 最好使 $Q[e]$ 为 $0$,从而 $R[f]=I[e]$ 可直接算出
    \footnote{若 $Q[f]$ 的形式对称,则可以通过取若干对称的点实现这一点,参见下面的例子(\arabic{chapter}.\ref{6-ex1})。}
    。
\end{itemize}

\example\label{6-ex1} 用广义 Peano 定理估计 Simpson 公式
\[ \int_{-1}^1f(x)\di x\approx Q[f]=\frac13[f(-1)+4f(0)+f(1)] \]
的误差,可考虑取 $-1,0,0,1$ 四点(其中 $0$ 为\emph{重节点}),代入 (\ref{6-e2}) 式即得
\[ e(x)=\frac{f^{(4)}(\xi)}{4!}(x+1)x^2(x-1) \]
可以算得 $Q[e]=0$,故直接有
\[\begin{aligned}
    R[f]&=I[e]=\int_{-1}^1\frac{f^{(4)}(\xi)}{4!}(x+1)(x-1)x^2\di x\\
    &=\frac1{90}f^{(4)}(\eta)\,(\eta\in[a,b]) 
\end{aligned}\]
其中,在 $(x+1)(x-1)x^2$ 不变号的前提下应用了第一积分中值定理。

\entry \setkey{简化解法}{24K金法}:设数值积分公式 $Q[f]$ 的代数精度为 $m$,据广义 Peano 定理知其误差项
一定为 $Kf^{(m+1)}(\eta)$ 的形式;代入 $f(x)=x^{m+1}$ 即有
\[ K\cdot m! = R[f] = I[f] - Q[f] \]
将 $I[f]$ 与 $Q[f]$ 算出,即可求得系数 $K$。
(当 $m=4$ 时,方程左侧为 $24K$,故戏称此法为「24K金法」。)

\example 用「24K金法」估计 Simpson 公式的误差。(答案与例 \arabic{chapter}.\ref{6-ex1} 相同。)

\example 用「24K金法」估计积分公式
\[ \int_{-1}^1f(x)\di x\approx Q[f]=\frac43f\left(-\frac12\right)-\frac23f(0)+
\frac43f\left(\frac12\right) \]
的误差。(答案:$R[f]=\frac7{720}f^{(4)}(\eta)$。)

\entry 在一般的积分公式 $Q[f]=\dsum_{i=0}^nA_if(x_i)$ 中,全部的待定参数共 $2n+2$ 个,
可将它们全部通过待定系数法解出(而不必预先给定);因此,对由 $n+1$ 个节点确定的积分公式,
可以期望其\emph{最高具有 $m=2n+1$ 的代数精度},进而列出 $m+1=2n+2$ 个方程将待定参数解出。

\entry 对代数精度为 $2n+1$ 的积分公式 $Q[f]=\dsum_{i=0}^nA_if(x_i)$,
可代入以下的 $2n+2$ 次多项式
\[ p(x)=(x-x_0)^2(x-x_1)^2\cdots(x-x_n)^2 \]
则有 $Q[p(x)]=\dsum_{i=0}^nA_ip(x_i)=0$,而 $I[p(x)]$ 必然为正数,故可见 $Q[f]$ 的代数精度\emph{不可能}达到 $2n+2$,任何情况下的最高代数精度只能是 $2n+1$。

\example 求解使积分公式
\[ \int_{-1}^1f(x)\di x=Q[f]=A_0f(x_0)+A_1f(x_1) \]
代数精度尽可能高的 $A_0,A_1,x_0,x_1$,并估计误差。
(答案:$Q[f]=f(-1/\sqrt3)+f(1/\sqrt3)$,$m=3$,$R[f]=\frac1{135}f^{(4)}(\eta)$.)

\entry \key{Gauss 求积公式}:具有 $n+1$ 个节点,代数精度达到 $2n+1$ 的数值积分公式。相应的节点称为 \key{Gauss 点}。

\trm 求积公式 $Q[f]=\dsum_{i=0}^nA_if(x_i)$ 是 Gauss 求积公式的\setkey{充要条件}{Gauss 求积公式充要条件}为:
\begin{itemize}\tl
    \item $x_i$ 为 $[a,b]$ 上关于权系数 $\omega(x)$ 正交的多项式 $g_{n+1}(x)$ 之零点;
    \item 求积系数 $A_i$ 按下式确定:
    \begin{equation}\label{6-e5}
    A_i=\int_a^b\omega(x)l_i(x)\di x
    \end{equation}
\end{itemize}

\trm 引理:设$\{g_k(x)\}$ 为 $[a,b]$ 上关于 $\omega(x)$ 正交的\emph{首一正交多项式},则有
\begin{equation}
\frac{g_{k+1}(x)}{x-x_i}=\frac{\gamma_n}{g_n(x_i)}\sum_{k=0}^n\frac{g_k(x)
g_n(x_i)}{\gamma_k}
\end{equation}
其中 $x_i\ (i=0,1,\ldots,n)$ 为 $g_{n+1}$ 的零点,$\gamma_k=(g_k,g_k)$ 与正交多项式\emph{三项递推关系}(条目 5.\ref{5-t1})中的含义一致
\footnote{推导也须采用 5.\ref{5-t1} 中的三项递推关系:对其中 $g_{k+1}(x)$ 的表达式进行若干变换,再对 $k$ 求和就可得到此式。}
。

\trm \key{Gauss 求积公式系数公式}:给定一组首一正交多项式$\{g_k(x)\}$,则 Gauss 求积公式 $Q[f]=\dsum_{i=0}^nA_if(x_i)$ 中的系数 $A_i$ 可按
\begin{equation}\label{6-e3}
A_i=\frac{\gamma_n}{g'_{n+1}(x_i)g_n(x_i)}\quad(i=0,1,\ldots,n)
\end{equation}
计算
\footnote{证明思路:根据 $x_i$ 为 $g_{n+1}(x)$ 的条件,将式 \eqref{6-e5} 中的 Langrange 插值基函数变换为 $l_i(x)=\frac{g_{n+1}(x)}{(x-x_i)g'_{n+1}(x_i)}$ 即可。}
。

\trm \key{Gauss 求积公式截断误差}:若 $f\in C^{2n+2}[a,b]$,且 $g_{n+1}(x)$ 为一个首一多项式,则有
\begin{equation}\label{6-e4}
R[f]=\frac{\gamma_{n+1}}{(2n+2)!}f^{(2n+2)}(\eta).
\end{equation}

\trm Gauss 型求积公式的系数全大于 $0$。
\footnote{证明:对 $f(x)=l_k^2(x)$ 应用 Gauss 求积公式。与之相反,Newton-Cotes 公式的系数无法满足此种条件。}

\entry Gauss 型求积公式的\emph{求解过程}:
\begin{enumerate}\tl
    \item 按三项递推关系 (\ref{5-e1}) 求出在 $[a,b]$ 上关于 $\omega(x)$ 正交的
    多项式 $g_{n+1}(x)$;
    \item 求出 $g_{n+1}(x)$ 的 $n+1$ 个根 $x_0,x_1,\cdots,x_n$。
    \item 按式 (\ref{6-e3}) 求解积分公式中的系数。
    \item 按 (\ref{6-e4}) 式的结果求解误差。
\end{enumerate}

\entry Gauss 求积公式的\emph{稳定性}:设计算函数值 $f(x_i)$ 时的舍入误差可表示为
\[ |f(x_i)-\tilde{f}(x_i)|\leq\vepsilon \]
其中 $\vepsilon$ 为各子区间上\emph{最大}的舍入误差限,则总的舍入误差估计为
\begin{equation}
E=\left|\sum_{i=0}^nA_if(x_i)-\sum_{i=0}^nA_i\tilde{f}(x_i)\right|\leq\sum_{i=0}%
^n\bigl|A_i\bigr|\vepsilon.
\end{equation}
对 Gauss 求积公式,所有的 $A_i$ 均为正,故有
\[ \sum_{i=0}^n\bigl|A_i\bigr|=\sum_{i=0}^nA_i=\int_a^b\omega(x)\di x=\gamma_0 \]
从而说明 $E\leq\gamma_0\vepsilon$,舍入误差可控。这说明 \emph{Gauss 求积公式是稳定的}。

\section{数值微分公式}
\entry 最基本的算法:\emph{割线代切线}
\[ f'(x_i)\approx\frac{f(x_{i+1})-f(x_i)}{x_{i+1}-x_i}=A_if(x_i)+A_{i+1}f(x_{i+1}) \]
可见数值微分公式与数值积分公式本质相同,均为个别点\emph{函数值的组合}。其缺陷与近似积分法(条目 \arabic{chapter}.\ref{6-et1})相同:\emph{误差无法估计}。为此,应直接从一般的公式入手,在给定条件下直接求解节点与系数,并估计误差。

\entry \key{插值型数值微分公式}:用插值多项式 $L_n(x)$ 代替原有函数 $f(x)$ ,求其导数:
\[ f(x)=L_n(x)+R_n(x)\,\Rightarrow\,f^{(k)}(x)=L_n^{(k)}(x)+R_n^{(k)}(x).\]

\entry 插值型公式的\emph{误差估计}:
\[ R_n^{(k)}(x)=\frac{\di^k}{\di x^k}\left[\frac{f^{(n+1)}(\xi)}{(n+1)!}
\pi_{n+1}(x)\right]=\frac{\di^k}{\di x^k}\bigl(f[x_0,x_1,\cdots,x_n,x]\pi_{n+1}(x)
\bigr) \]

\trm 差商的导数有如下性质:
\begin{equation}
\frac{\di^k}{\di x^k}\bigl(f[x_0,x_1,\cdots,x_m,x]\bigr)=(k!)\cdot f[x_0,x_1,\cdots,x_m,x,x,\cdots,x].
\end{equation}

\entry 两点数值微分公式($n=1,k=1$):对 $x_0,x_1,x$ 三点作 Newton 插值
\footnote{也可用 Lagrange 插值法,并直接使用 Lagrange 插值法的误差估计式。}
,利用差商导数的性质对插值多项式 $f(x)$ 求导,可以求得
\[ f'(x)=f[x_0,x_1] + f[x_0,x_1,x,x](x-x_0)(x-x_1)+ f[x_0,x_1,x](2x-x_0-x_1) \]
分别代入 $x_0$ 与 $x_1$ 可得两个数值微分公式:
\begin{gather}
f'(x_0)=\frac{f(x_1)-f(x_0)}{x_1-x_0}+\frac12f''(\xi_0)(x_0-x_1)\label{7-e3}\\
f'(x_1)=\frac{f(x_1)-f(x_0)}{x_1-x_0}+\frac12f''(\xi_1)(x_0-x_1)\label{7-e4}
\end{gather}
可用记号 $h=x_1-x_0$ 简化写法。

\entry 三点数值微分公式($n=2,k=1$):设三个等距(间距为 $h$)节点分别为 $x_0,x_1,x_2$,用 Newton 插值多项式可得到
\begin{gather}
f'(x_0)=\frac1{2h}[-3f(x_0)+4f(x_1)-f(x_1)]+\frac{h^2}3f'''(\xi_0)\\
f'(x_1)=\frac1{2h}[-f(x_0)+f(x_2)]-\frac{h^2}6f'''(\xi_1)\\
f'(x_2)=\frac1{2h}[f(x_0)-4f(x_1)+3f(x_2)]+\frac{h^2}3f'''(\xi_2).
\end{gather}

\entry 三点二阶数值微分公式($n=2,k=2$):在以上公式推导过程中再求一次导,分别代入三个
等距节点即得 
\begin{gather}
f''(x_0)=\frac1{h^2}[f(x_0)-2f(x_1)+f(x_2)]-hf'''(\xi_1)+\frac{h^2}6f^{(4)}(\xi_2)\\
f''(x_1)=\frac1{h^2}[f(x_2)-2f(x_1)+f(x_2)]-\frac{h^2}{12}f^{(4)}(\xi)
\label{7-e-mid}\\
f''(x_2)=\frac1{h^2}[f(x_0)-2f(x_1)+f(x_2)]+hf'''(\xi_1)+\frac{h^2}6f^{(4)}(\xi_2).
\end{gather}

\entry 一般,常用以下两个在区间中点$x_1$取得的数值微分公式:
\begin{gather}
f'(x)=\frac{f(x+h)-f(x-h)}{2h}-\frac{h^2}6f'''(\xi),\\
f''(x)=\frac{f(x+h)-2f(x)+f(x+h)}{h^2}-\frac{h^2}{12}f^{(4)}(\xi).
\end{gather}

\entry 数值微分的\setkey{待定系数法}{数值微分待定系数法}:类似于数值积分,用待定系数法追求更高的\emph{代数精度}
\footnote{此处,代数精度的定义与插值函数、数值积分中的相同。}
。

\example 给定 $x_0,x_1$,为求得下列形式的数值微分公式
\[ f''(x_0)\approx c_0f(x_0)+c_1f'(x_0)+c_2f(x_1) \]
中系数 $c_0,c_1,c_2$ 的值,首先期望该公式具有 $m=2$ 的代数精度
\footnote{此时刚好可列出三个方程解出三个系数。}
,进而可依次代入
$f(x)=1$、$f(x)=x$、$f(x)=x^2$:
\[\begin{cases}-c_0-c_2=0\\-c_1-c_2h=0\\2-c_2h^2=0\end{cases}\sothat
\begin{cases}c_0=-\frac{2}{h^2}\\c_1=-\frac2h\\c_2=\frac2{h^2}\end{cases}\]
可将 $f(x)=x^3$ 代入最终的公式中,解得 $R[f]=-2h\neq0$,故该公式的代数精度止于 $2$。
此处可用广义 Peano 定理,取$x_0,x_0,x_1$ 三个节点构筑 $e(x)$,从而
\[\begin{aligned}
R[f]&=R[e(x)]=e''(x_0)-\left[-\frac2{h^2}e(x_0)+\frac2he'(x_0)-\frac2{h^2}e(x_1)\right]\\
&=e''(x_0)=-\frac h3f'''(\xi).
\end{aligned}\]
(当然,也可以用「24K金法」,将 $f(x)=x^3$ 直接代入 $I[f]=R[f]$ 之中,
求得系数 $K=-\frac h3$。)

\entry \key{Richardson 外推法}:类似于 Romberg 积分法,用低精度公式递推出高精度公式。

\entry \key{变步长微分法}:类于数值积分中的类似方法。

\example 欲求得数值微分公式
\[ f''(x)\approx Af(x-h)+Bf(x)+Cf(x+h) \]
中的系数 $A,B,C$,可利用 \emph{Taylor 公式}展开等式右侧的 $f(x-h)$ 与 $f(x-h)$:
\[ f''(x)=(A+B+C)f(x)+(-A+C)hf'(x)+(A+C)\frac{h^2}2f''(x)+R(x) \]
其中 $R(x)$ 为余项。左右对应项相等,故有
\[
\begin{cases}A+B+C=0\\-A+C=0\\A+C=\frac2{h^2}\end{cases}\sothat
\begin{cases}A=-\frac1{h^2}\\B=-\frac2{h^2}\\C=\frac1{h^2}\end{cases}
\]
故可推出数值微分公式为
\[ D_h[f]=\frac1{h^2}[f(x-h)-2f(x)+f(x+h)] \]
为提高该公式的精度,可再取 $h/2$ 为步长
\footnote{从这里出发,就可以如 Romberg 积分法的推导过程一样,导出形式类似的 Richardson 外推法。详见李乃成、梅立泉《数值分析》第 205 -- 206 页6.4.3节「外推求导法」。}
,能递推出近似关系
\[ f''(x)-D_{\frac12h}[f]\approx\frac14(f''(x)-D_h[f])\sothat
f''(x)\approx\frac13\left[4D_{\frac h2}[f]-D_h[f]\right]. \]
记 $\overline{D}[f]=\frac43D_{\frac h2}[f]-\frac13D_h[f]$,该公式相较于上式更为精确。