\chapter{附录:考试内容评析}
\begin{itemize}
\item 在本校,与「数值计算方法」相关的课程大致可以分为工科生的\emph{计算方法}与数学系学生的\emph{数值分析}两大类。后者较前者难度更高,对计算与证明过程有更详细的考察;而目前流传的「往年试卷」中,往往以\emph{数值分析}的考卷居多,这会给修\emph{计算方法}课程的同学造成误导。因此,请不要过于相信这些「往年考卷」。
\item 关于\emph{复习}:复习过程中并不需要做太多的「练习」,只需牢记相关知识点和例题即可。仅就这份笔记而言,考试中所有可能出现的知识点均已涉及到了。
\item 关于\emph{考试题型}:从近几年的情况来看,一般分为\emph{填空题}和\emph{计算题}两大类,各占一半左右的分数。其中:
\begin{itemize}
    \item 填空题主要考察知识点,基本上不需要计算(口算就能解决)。大多数题目都有「窍门」,很多看似复杂的题目之结果其实非常简单。课程中所有的基础知识点都有可能涉及到。分值很高,注意不要丢分。
    \item 计算题更像是「证明题」或「简答题」,主要目的在于考察学生对各类计算方法原理的理解应用能力(主要)或推导证明能力(次要)。具体的数值计算量并不大。
\end{itemize}
\item 计算题常见考点:以下考点基本上是固定的,在作业题中也时常操练,不需要特别担心。
\begin{itemize}\tl
    \item 对矩阵做 LU 分解;
    \item 判断三种线性方程迭代法的收敛性(近年来较少考,但有可能出此类题目);
    \item 对给定数据点做 Newton 插值;
    \item 利用待定系数法推导简单的数值积分公式,使之达到指定代数精度;
    \item 判断非线性方程迭代格式的收敛性。
\end{itemize}
\end{itemize}
