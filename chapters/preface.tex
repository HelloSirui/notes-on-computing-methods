\chapter*{前言}
计算方法是本校(西安交通大学)钱学森班、少年班等拔尖班的同学必修的一门数学基础课程,其内容包括解决一些计算问题(如解线性方程组、插值与逼近、数值微积分等)的常用算法,及针对这些算法之误差与稳定性的相关数学理论。相较于其他的数学课程,计算方法课程具有内容丰富、实用性强等特点,但也兼有一般数学课程内容深、考点多的性质。可以说,计算方法课程是工科学生在走向专业课程之前所要翻越的最后一座数学理论「大山」。

2014级少年班的王天浩同学,于此方面做了相当好的工作,在学习本课程期间尽心尽力整理了十分详尽的课程笔记,基本上达到了以上所提到的要求。尽管该份笔记是纸质版扫描而成,但一经发布便广泛流传开来,成为当届及本届学生复习时重要的参考资料。为方便之后的同学复习计算方法课程,经与王天浩学长协商,我自2019年6月6日开始将原有的纸质扫描笔记以\LaTeX 整理为电子版,补足了语言上的一些省略、缺失,增写了许多注记、说明、插图,并扩充了一些新的条目。新的电子版本定名为《计算方法撷英》,以示此份文档与「计算方法」课程之关系。

\subsection*{正文格式说明}
电子版笔记基本上遵循了原有纸质笔记的框架,但相较与原来的样式更为清晰、简明。这份笔记的正文中包含了这样几类内容:
\begin{description}\tl
    \item[条目] 用 \ding{226} 符号引导的内容,附有编号,为一般性质的知识点、说明。
    \item[例题] 用 \ding{48} 符号引导的内容,附有编号,通常是对其正上方那一个或几个知识点的具体呈现与演示。一些给出详细过程,一些仅给出答案(供读者自行练习时参考)。
    \item[定理/定义] 用 \ding{43} 符号引导的内容,附有编号,表示一些比较重要的定理和概念定义。
    \item[关键词] 用\textbf{粗体}标注的内容,表示比较关键的概念。可在本笔记后的「索引」栏目查找本笔记中所有的关键词。 
    \item[强调] 用\emph{下划线}标注的内容,表示强调,以与其他内容区分开来。 
    \item[注记] 用脚注的方式给出,通常是对正文内容的进一步阐释,或对正文中略去之内容的说明。
\end{description}

\subsection*{撰写说明}
此份笔记自2019.6.6开始撰写,期间编者尚在美国交流,只能利用课余时间零碎增补;至9月回国时,正文内容几近完成。在编者缓考完成后,对内容的润色工作暂时搁置,直至11月时才重新启动,并最终完成。在编者整理稿件的过程中,除以王天浩同学的纸质笔记为底稿之外,也在获得许可的情况下参考了 2017 级钱班吴思源同学的计算方法笔记,在此向二位深表感谢!

在此之外,编者还要向授课的马军老师,及编撰本校计算方法相关教材的李乃成、邓建中、梅立泉等诸位老师表示敬意。作为一门与科技前沿紧密相关的数学基础课,授课者们担负的责任异常重大,而他们的表现则异常令人钦佩!

\subsection*{帮助我们改进这份笔记}
一本好的教科书,来自于相关教师历经数代、数十年的逐次再版改进;一份好的笔记,同样也需要长期的维护、改进才能够最终创造出来。本份笔记还未经这样久的磨砺,在内容、布局、细节等方面都相当欠缺;因此,恳请诸位读者在使用本份笔记时,留意以下几点:

\begin{itemize}\tl
    \item 检查正文中存在的笔误、错别字、公式错误等;
    \item 考察此份笔记是否缺少课程相关的知识点、章节内容;
    \item 评价本笔记中是否有详略不得体、例题过少、描述不明晰的内容。
\end{itemize}

以上三点,「境界」逐次提高,却都能够有效提高这份笔记的可用程度。若读者在阅读过程中发现以上三点问题,请将问题整理好,寄送到编者的电子邮箱:\url{yjr134@163.com};如您常年使用 GitHub,也可在本份笔记的开源仓库页面 \url{https://github.com/qyxf/notes-on-computing-methods} 上发布 issue 或 pull request,提出改进建议。

非常感谢各位读者的贡献。祝愿大家在考试中取得令人满意的成绩,并能够在实际问题中更为熟练的应用各类数值计算方法。

\begin{flushright}
能动少C71 尤佳睿
\footnote{个人博客:\url{https://www.cnblogs.com/xjtu-blacksmith/}。}
\\2019 年 11 月 24 日
\end{flushright}

\thispagestyle{plain}