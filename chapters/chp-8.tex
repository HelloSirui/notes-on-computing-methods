\chapter{常微分方程数值解}
\entry 本章只涉及\emph{一阶常微分方程初值问题}
\footnote{其他可能的问题包括:更高阶的常微分方程初值问题,以及常微分方程\emph{边值问题}等。}
:
\begin{equation}\label{8-e1}
\begin{cases}
y'(x)=f(x,y(x)),\quad x\in[a,b]\\
y(a)=y_0.
\end{cases}
\end{equation}

\trm \setkey{解的存在唯一性}{常微分方程解存在唯一性}:在由式 (\ref{8-e1}) 所述的初值问题中,若 $f(x,y)$ 在区域
\[D=\{(x,y)|a\leq x\leq b,-\infty<y<+\infty\}\]
中连续,且存在一个常数 $L$ 使得对任意选取的 $y$ 与 $\bar{y}$ 都有
\footnote{不等式中的第一个不等号,来自于多元函数的\emph{拟微分中值定理}。}
\[ |f(x,y)-f(x,\bar{y})|\leq\left|\frac{\del f}{\del y}(y-\bar{y})\right|\leq L|y-\bar{y}|,
\quad x\in[a,b] \]
则该初值问题在 $[a,b]$ 上存在唯一连续可微解 $y=y(x)$。

\entry \emph{数值解}:用数值计算方法,求出初值问题 (\ref{8-e1}) 在给定的\emph{若干节点} $x_i\in[a,b]$ 上的解 $y_i\approx y(x_i)$。

\entry 一般记 $x_i$ 处的\emph{精确解}为 $y(x_i)$,\emph{数值近似解}则记为
$y_i$。

\section{常用解法的导出}
\entry \key{数值微分法}:在初值问题 (\ref{8-e1}) 中,用某一数值微分公式
\[ y'(x_i)=y'_i+R[y] \]
(其中 $y'_i$ 为近似解,$R[y]$ 为误差项)替换原方程中的 $y'(x_i)$,进而求出一个求解 $y_i$ 的数值公式,误差估计由 $R[y]$ 给出。

\entry \key{Euler 法}:在 (\ref{8-e1}) 中代入 $x=x_i$,并利用 $x_i$ 与 $x_{i+1}$ 间的两点数值微分公式 \eqref{7-e3} 替换 $y'(x_i)$,可得到关系式
\[ \frac{y(x_{i+1})-y(x_i)}{h}-\frac12y''(\xi_i)h=f(x_i,y(x_i)) \]
作局部化假设 $y(x_i)=y_i$ (即认为 $x_i$ 处的解已准确算出),并略去关于 $h$ 的高阶项,
则可最终获得关于 $y_{i+1}$ 的递推关系
\begin{equation}
y_{i+1}=y_i+hf(x_i,y_i)
\end{equation}
此法称为 \emph{Euler 法}。

\entry \key{局部截断误差}:根据两点数值微分公式的误差,可知 Euler 法的截断误差为
\begin{equation}
R[x_i]=\frac{h^2}2y''(\xi_i).
\end{equation}

\entry \key{全局截断误差}:在\emph{局部化假设}处处成立
\footnote{即每一步所用的前提条件都准确无误。}
之前提下,全局误差可估计为
\begin{equation}
R(x_N)=\sum_{i=0}^{N-1}R[x_i]=\frac{b-a}2\cdot hf''(\xi).
\end{equation}

\entry \key{后退 Euler 法}:在初值问题中代入 $x=x_{i+1}$,并利用 $x_i$ 与 $x_{i+1}$ 间的两点数值微分公式 \eqref{7-e4} 替换 $y'(x_i)$,可得到关系式
\[ \frac{y(x_{i+1})-y(x_i)}{h}-\frac12y''(\xi_{i})h=f(x_{i+1},y(x_{i+1})) \]
作局部化假设 $y_{i+1}=y(x_{i+1})$ 并略去关于 $h$ 的高阶项,可获得关系式
\begin{equation}
y_{i+1}=y_i+hf(x_{i+1},y_{i+1})
\end{equation}
此法称为\emph{后退 Euler 法}。

\entry 后退 Euler 法中,$y_{i+1}$ 不能\emph{显式}解出,在 $f$ 为非线性函数时必须采用\emph{迭代解法},称该法为\key{隐式解法};Euler 法中可直接由 $x_i$ 解出 $y_{i+1}$,故称为\key{显式解法}。

\entry 后退 Euler 法的\emph{局部截断误差}为
\begin{equation}
R[x_i]=-\frac{h^2}2y''(\xi_i),
\end{equation}
与 Euler 法的符号相反。

\entry \key{中点法}:在初值问题中代入 $x=x_{i+1}$,并代入中点处的三点微分中值公式 \eqref{7-e-mid},可得到关系式
\[ \frac{y(x_{i+1})-y(x_{i-1})}{2h}-\frac{h^2}6y'''(\xi_i)=f(x_i,y(x_i)) \]
作局部化假设 $y_{i+1}=y(x_{i+1})$并略去关于 $h$ 的高阶项,可获得关系式
\begin{equation}
y_{i+1}=y_{i-1}+2hf(x_i,y_i)
\end{equation}
此法称为\emph{中点法}。

\entry 中点法中,应用了 $x_{i-1}$ 与 $x_i$ 两个前提条件,为此需在初值条件 $y_0$ 的基础上额外补充\emph{表头元素} $y_1$,为此称中点法为一个\key{多步法}。与此相对,两种 Euler 法都是\key{单步法}。

\entry 多步法的表头元素,常用单步法提供。

\entry 一般不采用点数更多的数值微分公式:精度提高不明显,且需要补充更多表头元素;精度受表头元素的制约。

\define 公式的\setkey{精度}{常微分方程数值解法精度}:若某一微分方程的数值解法的局部截断误差记为 $R[x_i]$,其满足
\[ R[x_i]=y(x_{i+1})-y_{i+1}=O(h^{p+1}) \]
则称该解法(公式、方法)的精度为 $p$ 阶。

\entry 两种 Euler 法的精度均为一阶,中点法为二阶。

\entry 数值积分法:对微分方程初值问题 (\ref{8-e1}) 作积分,得到同解的\emph{积分方程问题}
\begin{equation}
y(x)=y(a)+\int_a^xf(x,y(x))\di x\quad(x\in[a,b])\\
\end{equation}
在问题中替换区间 $[a,b]$ 为$[x_i,x_{i+1}]$,再利用\emph{数值积分公式}替代方程中的积分表达式,求得关于 $y_{i+1}$ 的关系式。

\entry 对子区间上积分 $\int_{x_i}^{x_{i+1}}f(x,y(x))\di x$,采用
\begin{itemize}
    \item \emph{矩形数值积分公式}:可分别获得数值解法
    \begin{gather}
    y_{i+1}=y_i+hf(x_i,y_i)\\
    y_{i+1}=y_i+hf(x_{i+1},y_{i+1})
    \end{gather}
    此即 Euler 法与倒退 Euler 法,均为一阶公式。
    \item \emph{梯形公式}:可获得数值解法
    \begin{equation}
    y_{i+1}=y_i+\frac h2[f(x_i,y_i)+f(x_{i+1},y_{i+1})]
    \end{equation}
    此公式也称\key{梯形公式},是一个\emph{二阶隐式公式},误差估计为
    \begin{equation}
    R[x_i]=-\frac{h^3}{12}f'''(\xi_i).
    \end{equation}
    \item \emph{Simpson 公式}:可获得数值解法
    \begin{equation}
    y_{i+1}=y_{i-1}+\frac h3[f(x_{i-1},y_{i-1})+4f(x_i,y_i)+f(x_{i+1},y_{i+1})]
    \end{equation}
    此公式也称为 \key{Simpson 公式},是一个\emph{四阶两步隐式公式},误差估计为
    \begin{equation}
    R[x_i]=-\frac{h^5}{90}y^{(5)}(\xi_i).
    \end{equation}
\end{itemize}

\entry \key{Adams 显式公式}:为求 $y_{i+1}$ 的值,给定之前的 $(x_{i-k},f_{i-k}),(x_{i-k+1},f_{i-k+1}),$ $\cdots,(x_i,f_i)$ 共 $(k+1)$ 个点(其中 $f_n=f(x_n,y_n)$),作\emph{插值多项式}
\[ f(x,y)=L_k(x)+R_k(x) \]
以获得 $y'(x)$ 的近似表示,进而在同解积分问题中近似代入
\[ \int_{x_i}^{x_{i+1}}y'(x)\di x\approx\int_{x_i}^{x_{i+1}}L_k(x)\di x \]
从而可以获得一显式的数值解公式
\begin{gather}
y_{i+1}=y_i+\frac hA\cdot(b_0f_i+b_1f_{i-1}+\cdots+b_kf_{i-k})\quad(i\geq k)\\
R[x_i]=B_kh^{k+2}y^{(k+2)}(\xi_i)
\end{gather}
称为 \emph{Adams 显式公式}。式中的系数 $A$、$f_i$、$B_k$ 可据实推导,亦可直接查现成的系数表
\footnote{系数表参见李乃成、梅立泉《数值分析》第269页表9.1。}
。

\entry Adams 隐式公式:与 Adams 显式公式推导类似,但在给定的 $k+1$ 数据点中,将最前的 $(x_{i-k},y_{i-k})$ 替换为待求的 $(x_{i+1},y_{i+1})$,仍作插值多项式,进而可求解得到一隐式的数值解公式
\begin{gather}
y_{i+1}=y_i+\frac h{A^\ast}\cdot(b^\ast_0f_{i+1}+b^\ast_1f_i+\cdots+b^\ast_kf_{i
-k+1})\quad(i\geq k)\\
R[x_i]=B^\ast_kh^{k+2}y^{(k+2)}(\xi_i)
\end{gather}
称为 \emph{Adams 隐式公式}。式中系数可据实推导,也可直接查表
\footnote{系数表参见李乃成、梅立泉《数值分析》第270页表9.2。}
。

\section{预测-校正方法与一般性理论}
\entry 一般而言,显式法\emph{易于计算},但隐式法的\emph{稳定性}较显式法高。

\entry 可以考虑在求解过程中,先用(低阶)显式公式初步求得 $y_{i+1}$ 的近似值,再直接代入(高阶)隐式公式中以直接求得更稳定的解。此类方法统称\key{预估-校正法}。

\entry \key{改进 Euler 法}:用 Euler 法预估,再用倒退 Euler 法校正:
\begin{equation}
\begin{cases}
p_{i+1}=y_i+h(x_i,y_i)\\
y_{i+1}=y_i+h(x_{i+1},p_{i+1})
\end{cases}.
\end{equation}
此法仍是一阶解法,但稳定性比 Euler 法更高,也不用像倒退 Euler 法一样隐式求解。

\entry \key{Heun 方法}:用 Euler 法预估,用梯形公式校正:
\begin{equation}
\begin{cases}
p_{i+1}=y_i+h(x_i,y_i)\\
y_{i+1}=y_i+\frac h2[f(x_i,y_i)+f(x_{i+1},p_{i+1})]
\end{cases}
\end{equation}
此法是二阶方法。

\entry 为考虑更一般的情形,可将 Heun 方法改写为
\begin{equation}
\begin{cases}
y_{i+1}=y_i+\frac12K_1+\frac12K_2\\
K_1=hf(x_i,y_i)\\
K_2=hf(x_i+h,y_i+K_1)
\end{cases}
\end{equation}
式中,$K_1$ 与 $K_2$ 是利用已有的 $(x_i,y_i)$ 信息依次用 $f$ 推演所得的\emph{补充信息}。

\entry \key{Runge-Kutta 法}(RK 法):将以上所言的「补充信息」一般化,可得如下的解法:
\begin{equation}\label{8-e3}
\begin{cases}
y_{i+1}&=y_i+\lambda_1K_1+\lambda_2K_2+\cdots+\lambda_mK_m\\
K_1&=hf(x_i,y_i)\\
K_2&=hf(x_i+\alpha_2h,y_i+\beta_{21}K_1)\\
K_3&=hf(x_i+\alpha_3h,y_i+\beta_{31}K_1+\beta_{32}K_2)\\
\vdots&\\
K_m&=hf(x_i+\alpha_mh,y_i+\beta_{m1}K_1+\cdots+\beta_{m,m-1}K_{m-1})
\end{cases}
\end{equation}
式中的 $\lambda_k,\alpha_k,\beta_{ij}$等系数待定。为追求 $m$ 阶的精度,即达到
\begin{equation}
R[x_i]=o(h^{m+1})
\end{equation}
的截断误差,可将 $R[x_i]$ 的表达式在 $x_i$ 处\emph{展开},使其低于 $m+1$ 次各项系数为 $0$,进而列方程确定式 \eqref{8-e3} 中各项系数。此类方法统称为 \emph{Runge-Kutta} 法。

\entry 实际问题中,方程数往往小于系数个数,此时可自由选取个别系数,以确定其他系数。
(由此得到的是不同的解法。)

\example 二阶 Runge-Kutta 法:改进 Euler 法,变形 Euler 法
\footnote{参见李乃成、梅立泉《数值分析》第275页「变形欧拉法」。}
。

\example 四阶 Runge-Kutta 法:最常用者为标准四级四阶 R-K 法:
\begin{equation}
\begin{cases}
y_{i+1}&=y_i+\frac16(K_1+2K_2+2K_3+K_4)\\
K_1&=hf(x_i,y_i)\\
K_2&=hf\left(x_i+\frac12h,y_i+\frac12K_1\right)\\
K_3&=hf\left(x_i+\frac12h,y_i+\frac12K_2\right)\\
K_4&=hf(x_i+h,y_i+K_3)
\end{cases}
\end{equation}

\entry Runge-Kutta 是单步显式方法,可直接求解,精度很高。缺点是计算量大。

\entry 可将之前所提的所有解法\emph{统一}成以下的形式:
\begin{equation}\label{8-e2}
y_{i+1}=\sum_{j=0}^K\alpha_jy_{i-j}+h\sum_{j=-1}^K\beta_jf_{i-j}.
\end{equation}
当 $K=0$ 时,其表示了一个单步法,否则表示了一个多步法。当 $\beta_{-1}=0$ 时,其表示了
一个显式公式,否则表示的是隐式公式。含 $y_{i-j}$ 的项决定了方程的步数,而含 $f_{i-j}$
的项决定了方程的阶数。

\entry \setkey{待定系数法}{常微分方程数值解待定系数法}:对由 (\ref{8-e2}) 概括的数值解法,可计算其截断误差
\begin{equation}
R[x_i]=y(x_{i+1})-y_{i+1}=y(x_{i+1})-\sum_{j=0}^K\alpha_jy(x_{i-j})-h\sum_{j=-1}^K\beta_jy'(x_{i-j}).
\end{equation}
利用 Taylor 公式将 $R[x_i]$ 中各项在 $x_i$ 处展开,为尽可能地使方法有高的精度,设其各
低次项的系数为 $0$,由此即可列解系数方程。进而可用广义 Peano 定理或「24K金法」求解截断
误差的系数。

\example 为确定所有可能的\emph{三阶两步方法},根据一般形式 (\ref{8-e2}) 可设出如下形式的解法:
\[ y_{i+1}=\alpha_0y_i+\alpha_1y_{i-1}+h(\beta_{-1}f_{i+1}+\beta_0f_i+\beta_1
f_{i-1}) \]
计算其截断误差为
\[
R[x_i] = y(x_{i+1}) - \alpha_0y(x_i) - \alpha_1y(x_{i-1}) -h[\beta_{-1}y'(x_i+h)+\beta_0y'(x_i)+\beta_1y'(x_i-h)]
\]
取 $x_i=0$,并令 $ R[x^k] = 0\ (k=0,1,2,3)$(3 阶精度),可得
\[
\begin{cases}
1-\alpha_0-\alpha_1=0\\
h[1+\alpha_1-(\beta_{-1}+\beta_0+\beta_1)]=0\\
h^2[1-\alpha_1-2(\beta_{-1}-\beta_1)]=0\\
h^3[1+\alpha_1-3(\beta_{-1}+\beta_1)]=0
\end{cases}
\]
此时有 $4$ 个方程 $5$ 个未知数,其中一个未知数可任意决定。设 $\alpha_1$ 任意,则可将其他 $4$ 个未知数表示为:
\[\begin{cases}
\alpha_0=1-\alpha_1\\
\beta_{-1}=\frac{5-\alpha_1}{12}\\
\beta_0=\frac{2+2\alpha_1}3\\
\beta_1=\frac{5\alpha_1-1}{12}
\end{cases}\]
考虑 $\alpha_1$ 的不同取值:
\begin{enumerate}
\item 若取 $\alpha_1=0$,则有 $\alpha_0=1$,$\beta_{-1}=\frac5{12}$,$\beta_0=\frac23$,$\beta_1=-\frac1{12}$;此即 $k=2$ 的 \emph{Adams 隐式公式}。
\item 若取 $\alpha_1=1$,则有 $\alpha_1=0$,$\beta_{-1}=\beta_1=\frac13$,$\beta_0=\frac43$;此即 \emph{Simpson 公式},其精度事实上为 4 阶。
\item 也可以取 $\alpha_1$ 为其他值,得到不同的公式。
\end{enumerate}

\entry 恭喜你把笔记看完了,请喝口水休息一下。