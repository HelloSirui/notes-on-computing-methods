\chapter{线性方程组迭代解法}
\section{迭代方法概要}


\entry 思想:$f(x^{\ast})=0\sothat\cdots\sothat x^{\ast}=\phi(x^{\ast})$,给出初值$x_0$和递推公式$x_{k+1}=\phi(x_k)$,假设$\{x_k\}$收敛,求极限——设$\lim\limits_{k\to\infty}x_k=x$,则有$x=\phi(x)$,从而必有$f(x)=0$。

\define \key{向量序列收敛}:$\mathbf{x}^{(k)}=(x_1^{(k)},x_2^{(k)},\cdots,x_n^{(k)})^{\mathrm{T}}$,若$\lim\limits_{k\to\infty}x_i^{(k)}=x_i^{\ast}$,则$\mathbf{x}^{(k)}\to\mathbf{x}^{\ast}\ (k\to\infty)$,记作$\lim\limits_{k\to\infty}\mathbf{x}^{(k)}=\mathbf{x}^{\ast}$。

\define \key{矩阵序列收敛}:$\mathbf{A}^{(k)}=(a_{ij}^{(k)})_{m\times n}$,若$\lim\limits_{k\to\infty}a_{ij}^{(k)}=a_{ij}$,则称$\mathbf{A}^{(k)}\to\mathbf{A}=(a_{ij})_{m\times n}$记作$\lim\limits_{k\to\infty}\mathbf{A}^{(k)}=\mathbf{A}$。
\footnote{此处不用$\mathbf{A}^\ast$,以免与伴随矩阵的符号$\mathbf{A}^\ast$弄混。}

\trm \key{序列收敛定理}:
\begin{itemize}\tl
    \item 对向量,$\mathbf{x}^{(k)}\to\mathbf{x}^\ast\ \Leftrightarrow\ \lim\limits_{k\to\infty}\|\mathbf{x}^\ast-\mathbf{x}^{(k)}\|=0$
    \item 对矩阵,$\mathbf{A}^{(k)}\to\mathbf{A}\ \Leftrightarrow\ \lim\limits_{k\to\infty}\|\mathbf{A}-\mathbf{A}^{(k)}\|=0$
\end{itemize}

\trm 定理:设$\mathbf{B}\in\mathbb{R}^{m\times n}$,则$\lim\limits_{k\to\infty}\mathbf{B}^k=0\ \Leftrightarrow\ \rho(\mathbf{B})<1$。

\entry \key{迭代格式}的构造:设$\mathbf{Ax=b}$,\emph{同解变形}得$\mathbf{x=Bx+g}$,可构造对应的\emph{迭代格式}:
\begin{equation}\label{3-e1}
    \mathbf{x}^{(k+1)}=\mathbf{Bx}^{(k)}+\mathbf{g}
\end{equation}
由于是同解变形,故$\mathbf{x}^\ast=\lim\limits_{k\to\infty}\mathbf{x}^{(k)}$满足$\mathbf{x}^\ast=\mathbf{Bx}^\ast+\mathbf{g}\sothat\mathbf{Ax}^\ast=\mathbf{b}$\ ,从而可通过(\ref{3-e1})式不断迭代以逼近方程的解。


\section{三种基本迭代法}

\entry \key{Jacobi迭代法}:以第$i$行为例,有:
\[a_{i1}x_1+a_{i2}x_2+\cdots+a_{ii}x_i+\cdots+a_{in}x_n=b_i\]
可解出
\begin{align}
x_i&=\frac1{a_{ii}}[b_i-(a_{i1}x_1+a_{i2}x_2+\cdots+a_{i,i-1}x_{i-1}+a_{i,i+1}x_{i+1}+\cdots+a_{in}x_n)]\notag\\
&=\frac1{a_{ii}}\left[b_i-\sum_{j=1,j\neq i}^na_{ij}x_j\right]
\end{align}
依上式即可构造Jacobi迭代格式:
\begin{equation}
x_i^{(k+1)}=\frac1{a_{ii}}\left[b_i-\sum_{j=1,j\neq i}^na_{ij}x_j^{(k)}\right]\quad(i=1,2,\cdots,n)
\end{equation}
按照$\mathbf{x=Bx+g}$的标准格式,整理成矩阵格式:
\begin{equation}
\begin{pmatrix}x_1\\x_2\\\vdots\\x_n\end{pmatrix}=\begin{pmatrix}0&-\frac{a_{12}}{a_{11}}&-\frac{a_{13}}{a_{11}}&\cdots&-\frac{a_{1n}}{a_{11}}\\-\frac{a_{21}}{a_{22}}&0&-\frac{a_{23}}{a_{22}}&\cdots&-\frac{a_{2n}}{a_{22}}\\\vdots&\vdots&\vdots&\ddots&\vdots\\-\frac{a_{n1}}{a_{nn}}&-\frac{a_{n2}}{a_{nn}}&-\frac{a_{n3}}{a_{nn}}&\cdots&0\end{pmatrix}\begin{pmatrix}x_1\\x_2\\\vdots\\x_n\end{pmatrix}+\begin{pmatrix}\frac{b_1}{a_{11}}\\\frac{b_2}{a_{22}}\\\vdots\\\frac{b_n}{a_{nn}}\end{pmatrix}
\end{equation}

\entry \key{Gauss-Seidel迭代法}:
\begin{equation}
\left\{
\begin{aligned}
x_1^{(k+1)}&=\frac1{a_{11}}(b_1-a_{12}x_2^{(k)}-a_{13}x_3^{(k)}-\cdots-a_{1n}x_n^{(k)})\\
x_2^{(k+1)}&=\frac1{a_{22}}(b_2-a_{21}\fbox{$x_1^{(k+1)}$}-a_{23}x_3^{(k)}-\cdots-a_{2n}x_n^{(k)})\\
x_3^{(k+1)}&=\frac1{a_{33}}(b_2-a_{31}\fbox{$x_1^{(k+1)}$}-a_{32}\fbox{$x_2^{(k+1)}$}-\cdots-a_{3n}x_n^{(k)})\\
\cdots&\cdots\\
x_n^{(k+1)}&=\frac1{a_{nn}}(b_2-a_{n1}\fbox{$x_1^{(k+1)}$}-a_{n2}\fbox{$x_2^{(k+1)}$}-\cdots-a_{n,n-1}\fbox{$x_{n-1}^{(k+1)}$})
\end{aligned}
\right.
\end{equation}
形式与Jacobi法相同,但区别在于进行下一步变量的迭代时采用了「新解」(即上式中框出的部分)。

\entry \key{超松弛迭代法}(SOR
\footnote{Successive over-relaxation.}
法):对Gauss-Seidel法的通用格式进行改写;若记每次迭代时的误差为$\mathbf{r}^{(k+1)}$,即
\[x_i^{(k+1)}=x_i^{(k)}+\frac1{a_{ii}}\left[b_i-\sum_{j=1}^{i-1}a_{ij}x_j^{(k+1)}-\sum_{j=i}^na_{ij}x_j^{(k)}\right]=x_i^{(k)}+\frac{r_i^{(k+1)}}{a_{ii}}\]
当$k\to\infty$时总有$r_i^{(k+1)}/a_{ii}\to0$,故可以乘一个系数$\omega$以\emph{加快收敛}:
\[x_i^{(k+1)}=x_i^{(k)}+\frac\omega{a_{ii}}\left[b_i-\sum_{j=1}^{i-1}a_{ij}x_j^{(k+1)}-\sum_{j=i}^na_{ij}x_j^{(k)} \right] \]
整理得
\begin{equation}
x_i^{(k+1)}=(1-\omega)x_i^{(k)}+\frac\omega{a_{ii}}\left[b_i-\sum_{j=1}^{i-1}a_{ij}x_j^{(k+1)}-\sum_{j=i+1}^na_{ij}x_j^{(k)} \right]
\end{equation}
整体迭代格式:
\begin{equation}
\left\{
\begin{aligned}
x_1^{(k+1)}&=(1-\omega)x_1^{(k)}+\frac\omega{a_{11}}\left(b_1-a_{12}x_2^{(k)}-a_{13}x_3^{(k)}-\cdots-a_{1n}x_{11}^{(k)} \right)\\
x_2^{(k+1)}&=(1-\omega)x_2^{(k)}+\frac\omega{a_{11}}\left(b_2-a_{21}x_1^{(k+1)}-a_{23}x_3^{(k)}-\cdots-a_{2n}x_{11}^{(k)} \right)\\
\cdots&\cdots\\
x_n^{(k+1)}&=(1-\omega)x_n^{(k)}+\frac\omega{a_{n1}}\left(b_n-a_{n1}x_1^{(k+1)}-\cdots-a_{n,n-1}x_{11}^{(k+1)} \right)
\end{aligned}
\right.
\end{equation}

\entry 迭代的\emph{矩阵表示法}:将$\mathbf{A}$分解为$\mathbf{D,E,F}$三部分:$\mathbf{A=D-E-F}$,其中
\begin{equation}
\mathbf{D}=\begin{spmatrix}a_{11}&&\\&\ddots&\\&&a_{nn}\end{spmatrix},
\mathbf{E}=\begin{spmatrix}0&&&&&\\-a_{21}&0&&&\\-a_{31}&-a_{32}&0&&\\\vdots&\vdots&\vdots&\ddots&\\-a_{n1}&-a_{n2}&-a_{n3}&\cdots&0 \end{spmatrix},
\mathbf{F}=\begin{spmatrix}0&-a_{12}&-a_{13}&\cdots&-a_{1n}\\&0&-a_{23}&\cdots&-a_{2n}\\&&0&\cdots&-a_{3n}\\&&&\ddots&\vdots\\&&&&0 \end{spmatrix}
\end{equation}
\begin{itemize}\tl
    \item \emph{Jacobi迭代法}:迭代格式中除对角阵 $\mathbf{D}$ 以外的系数均挪到了方程右侧,相当于
    \[\mathbf{(D-E-F)x=b}\sothat\mathbf{Dx=(E+F)x+b}\sothat\mathbf{x=D}^{-1}\mathbf{(E+F)x+D}^{-1}\mathbf{b}\]
    可见有
    \begin{equation}
    \begin{cases}\mathbf{B=D}^{-1}\mathbf{(E+F)=D}^{-1}(\mathbf{D-A})=\mathbf{I-D}^{-1}\mathbf{A}\\\mathbf{g=D}^{-1}\mathbf{b}\end{cases}
    \end{equation}
    \item \emph{Gauss-Seidel迭代法}:与 Jacobi 法不同的是,迭代中表示预先求得之「新解」的系数阵 $\mathbf{E}$ 留在方程左侧,相当于
    \begin{align*}
    \mathbf{(D-E-F)x=b}&\sothat\mathbf{(D-E)x=Fx+b}\\&\sothat\mathbf{x=(D-E)}^{-1}\mathbf{Fx+(D-E)}^{-1}\mathbf{b}
    \end{align*}
    可见有
    \begin{equation}
    \begin{cases}
    \mathbf{B=(D-E)}^{-1}\mathbf{F}\\
    \mathbf{g=(D-E)}^{-1}\mathbf{b}
    \end{cases}
    \end{equation}
    \item \emph{超松弛迭代法}:在 Gauss-Seidel 法的基础上,将 $(1-\omega)$ 大小的对角系数 $\mathbf{D}$ 移至方程右侧(作为误差),相当于
    \begin{gather*}
    \omega\mathbf{(D-E-F)x}=\omega\mathbf{b}\sothat(\mathbf{D}-\omega\mathbf{E})\mathbf{x}=[(1-\omega)\mathbf{D}+\omega\mathbf{F}]\mathbf{x}+\omega\mathbf{b}\\
    \sothat\mathbf{x}=(\mathbf{D}-\omega\mathbf{E})^{-1}[(1-\omega)\mathbf{D}+\omega \mathbf{F}]\mathbf{x}+(\mathbf{D}-\omega\mathbf{E})^{-1}\omega\mathbf{b}
    \end{gather*}
    可见有
    \begin{equation}
    \begin{cases}
    \mathbf{B}=(\mathbf{D}-\omega\mathbf{R})^{-1}[(1-\omega)\mathbf{D}+\omega\mathbf{F}]\\
    \mathbf{g}=\omega(\mathbf{D}-\omega\mathbf{E})^{-1}\mathbf{b}
    \end{cases}
    \end{equation}
\end{itemize}



\section{迭代收敛理论}
\entry 下面给出迭代格式收敛的条件(非常重要!个人猜测必考
\footnote{王天浩的评论。}
)。

\trm 定理1:若$\|\mathbf{B}\|\leq1$,则$\forall\mathbf{x}^{(0)}$,迭代格式$\mathbf{x}^{(k+1)}=\mathbf{Bx}^{(k)}+\mathbf{g}$收敛于解$\mathbf{x}^\ast$,且有误差估计式:
\begin{gather}
\|\mathbf{x^\ast-x}^{(k)}\|\leq\frac{\|\mathbf{B}\|}{1-\|\mathbf{B}\|}\|\mathbf{x}^{(k)}-\mathbf{x}^{(k-1)}\|\quad\text{(\emph{事后估计}或\emph{后验估计})}\\
\|\mathbf{x^\ast-x}^{(k)}\|\leq\frac{\|\mathbf{B}\|^k}{1-\|\mathbf{B}\|}\|\mathbf{x}^{(1)}-\mathbf{x}^{(0)}\|\quad\text{(\emph{事前估计}或\emph{先验估计})}
\end{gather}

\entry 估计式的推导:利用迭代格式及 $\mathbf{x}^\ast=\mathbf{Bx}^\ast+\mathbf{g}$ 条件将 $\mathbf{x}^\ast-\mathbf{x}^{(k)}$ 展开并变形,最终在等式两侧取范数,应用范数若干性质和条目 2.\ref{I-B} 的结果获得不等号。

\trm 定理2:$\forall\mathbf{x}^{(0)}$,迭代格式$\mathbf{x}^{(k+1)}=\mathbf{Bx}^{(k)}+\mathbf{g}$收敛于解$\mathbf{x}^\ast$的充要条件是下列两条件之一成立:
\begin{enumerate}\tl
    \item $\mathbf{B}^k\to\mathbf{O}$;
    \item $\mathbf{B}$的谱半径$\rho(\mathbf{B})<1$。
\end{enumerate}

\trm 推论:SOR法收敛的必要条件是$0<\omega<2$。
\begin{gather*}
\|\mathbf{B}\|=\|(\mathbf{D}-\omega\mathbf{E})^{-1}\|\cdot\|(1-\omega)\mathbf{D}+\omega\mathbf{F}\|=\frac{\|(1-\omega)\mathbf{D}\|}{\|\mathbf{D}\|}=(1-\omega)^n\\
\rho(\mathbf{B})<1\sothat|\lambda_1\cdots\lambda_n|<1\sothat|1-\omega|<1\sothat0<\omega<2.
\end{gather*}

\trm 对于三种常用迭代法,有更为实用的结论:
\begin{itemize}\tl
    \item 引理:若$\mathbf{A}$是严格对角占优矩阵,$0\le\omega\leq1$且$\lambda\geq1$时,矩阵$(\lambda+\omega-1)\mathbf{D}-\lambda\omega\mathbf{E}-\omega\mathbf{F}$也是严格对角占优矩阵。
    \item 推论2:若$\mathbf{A}$是\emph{严格对角占优矩阵},则$\forall\mathbf{x}^{(0)}$,Jacobi法、G-S法、SOR法($0<\omega\leq1$)均收敛
    \footnote{推导:反设$\mathbf{B}$ 有一特征值 $|\lambda|\geq1$,代入上面的推论中,推出$|\lambda\mathbf{I-B}|\neq0$,故所有的 $|\lambda|<1$,进而 $\rho(\mathbf{B})<1$。参见李乃成、梅立泉《数值分析》第 88 页推论 3.3.2 的证明。}
    。
    \item 推论3:若$\mathbf{A}$为对称正定阵,则$\forall\mathbf{x}^{(0)}$,Jacobi法收敛的充要条件是:$2\mathbf{D-A}$也是对称正定阵。
    \item 推论4:若$\mathbf{A}$是对称正定阵,则$\forall\mathbf{x}^{(0)}$,SOR法收敛充要条件为$0<\omega<2$。
\end{itemize}

\entry 对不同情况下的审敛法做总结:
\begin{enumerate}\tl
    \item 对角占优矩阵$\mathbf{A}$:Jacobi法收敛,G-S法收敛,$0<\omega\leq1$时SOR法收敛。
    \item 对称正定阵$\mathbf{A}$:
    \begin{itemize}\tl
        \item Jacobi法收敛$\ \Leftrightarrow\ 2\mathbf{D-A}$收敛;
        \item SOR法收敛$\ \Leftrightarrow\ 0<\omega<2$。
    \end{itemize}
    \item 一般矩阵$\mathbf{A}$:
    \begin{itemize}\tl
        \item $\|\mathbf{B}\|<1$时,$\mathbf{B}$ 的对应方法收敛;
        \item $\mathbf{B}^k\to\mathbf{O}$和$\rho(B)<1$中之一成立,则$\mathbf{B}$ 的对应方法收敛。
        \item SOR法收敛的必要条件是$0<\omega<2$。
    \end{itemize}
\end{enumerate}

\entry 针对考试而言,一般的系数矩阵仍需通过求 $\rho(\mathbf{B})$ 或 $\|\mathbf{B}\|$ 来判断收敛性。

\example 判断系数矩阵为$\mathbf{A}=\begin{spmatrix}2&3&4\\3&6&10\\4&10&20\end{spmatrix}$时各迭代法的收敛性。(答案:Jacobi 法发散,Gauss-Seidel 迭代法及 $0<\omega<2$ 的 SOR 法收敛。)
